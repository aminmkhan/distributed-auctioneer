
% Section: APPENDIX

\section*{Appendix}

\subsection{Proof of Theorem~\ref{theorem:simulation}}
\label{app:simulation}

\begin{proof}
First, we show that $P$ correctly simulates the auctioneer.
Every provider $j$ inputs $\vec{b}^j$ to the bid agreement.
By (1) of Property~\ref{prop:bid-agreement}, regardless of the inputs,
all providers output the same vector $\vec{b}$
that satisfies validity. By (1) of Property~\ref{prop:allocator},
the outcome of the simulation is pair $(x,\vec{p})$
with probability $\A(x,\vec{p} \mid \vec{b})$.
This concludes the first step of the proof.

Now, we show that $P$ is a $k$-resilient equilibrium for $m > f(k)$ for some $f$.
Fix a coalition $K$. We take $f$ to be larger for all $k$ than the minimum value of $m$
required by Properties~\ref{prop:bid-agreement} and~\ref{prop:allocator}.
These properties imply that, if providers have preference for a solution
at the bid agreement, then the implementations of bid agreement is $k$-resilient,
so providers in $K$ prefer to follow $P$ for bid agreement.
Since this guarantees that all providers have the same input at the
allocator, the implementation of the allocator is also $k$-resilient,
implying that $P$ is $k$-resilient.

Now, we show that solution preference holds for both blocks.
Recall that the outcome is not $\bot$ only if
providers not in $K$ output the same pair,
and, if the outcome is $\bot$, then the utility is $0$.
Hence, providers in $K$ prefer (obtain an expected utility at least as high)
that providers not in $K$ output the same pair $(x,\vec{p})$;
in this case, they clearly prefer to output $(x,\vec{p})$ as well, thus they have preference
for a solution at the allocator. Now, consider the bid agreement.
The utility of an outcome of this block is the expected
utility given that providers not in $K$ follow $P$ and providers in $K$
follow an arbitrary protocol. Clearly, providers in $K$
prefer that no provider in $K$ outputs $\bot$.
By (3) of Property~\ref{prop:allocator}, providers in $K$
prefer that all providers not in $K$ output the same vector.
By (2) of Property~\ref{prop:allocator},
providers in $K$ cannot increase the probability of any outcome of the framework 
other than $\bot$ by deviating, thus, they cannot increase their expected utility by outputting
a vector $\vec{b}' \neq \vec{b}$ at the bid agreement.
This shows that providers have preference for a 
solution at the bid agreement, concluding the proof.
\end{proof}


\subsection{Proof of Theorem~\ref{theorem:allocator}}
\label{app:allocator}

\begin{proof}
We show that $P$ ensures (1) correct simulation of $\A$; (2) resilience to collusive influence;
(3) input validation; and (4) $k$-resiliency for solution preference if all providers have the same input.
First, we show (1). Suppose that all providers input the same vector $\vec{b}$
and follow $P$. We show that every provider outputs the same pair $(x,\vec{p})$
with probability $\A(x,\vec{p} \mid \vec{b})$. We show using induction that,
if the decomposition of $\A$ into tasks is done correctly and we fix all random
numbers, then at every task $T$ every provider $j$ that executes $T$ has the same output
that he would have if $j$ executed $\A$ locally with the same random numbers.
This is true for the first task by (2) of Property~\ref{prop:iv}. In the inductive step,
the input at each task depends only on the output of a set of tasks.
For each of those tasks $T$, by the induction hypothesis,
a set $S$ of at least $k+1$ providers compute the 
same result and input it to the data transfer;
by (1) of Property~\ref{prop:data-transfer}, all providers that execute $T$
receive that value and perform the same computation as they would if they were executing $\A$.
This implies that all providers output the same pair at the end.
By (1) of Property~\ref{prop:common-coin}, at every invocation
of the common coin, all providers input the same distribution $\Pi$ and output the same random number
distributed according to $\Pi$, where $\Pi$ is specified by $\A$. This proves (1).

Now, we show (2). Fix a coalition $K$ and suppose that all providers 
not in $K$ follow $P$ with input $\vec{b}$ and providers in $K$ follow an arbitrary $P' \ne P$.
The only way that providers in $K$ could cause providers not in $K$
to return pair $(x,\vec{p})$ with probability higher than $\A(x,\vec{p} \mid \vec{b})$
is if the result of some task used in the input of another task or as the final output
is not distributed as specified by $\A$ and $\vec{b}$.
Since each task is executed by more than $k$ providers, 
using an identical reasoning to the proof of (1), we can show using induction
that providers in $K$ cannot manipulate the probability distribution over the results
of each task, except only by increasing the probability of some provider 
not in $K$ outputting $\bot$. Here, we use the fact that,
by (3) of Property~\ref{prop:iv} and
(2) of Properties~\ref{prop:common-coin},~\ref{prop:data-transfer},
providers in $K$ cannot manipulate the probability distribution over
outputs of the building blocks in a way that increases
the expected utility of some provider in $K$. This proves (2).

Condition (3) follows by (2) of Property~\ref{prop:iv}.
To show (4), we first need to prove that providers have preference for a solution at all invocations
of building blocks, assuming that they have preference for a solution of the allocator.
Fix a coalition $K$. It is clear that providers in $K$ prefer that providers not in $K$
do not output $\bot$ at all invocations. Now, we use backwards induction to show
that they prefer that providers not in $K$ never return different values.
In the last invocation, this is clearly true by preference for a solution of the allocator.
Continuing backwards, if two providers not in $K$ output different values
at the same invocation of some block, then either they output different pairs at the end or 
input different values at the following invocation of the data transfer,
which by the hypothesis is never preferable to outputting the same value at the considered invocation.
By the proof of (2), providers in $K$ cannot manipulate the final outcome
by not outputting the same values at all invocations, so
they also prefer to output the same values as providers not in $K$,
showing solution preference at all invocations. This also shows that
providers prefer to have the same input at all invocations.
Thus, given that all providers have the same input,
no provider in $K$ can increase its expected utility
if some provider $j \in K$ does not compute each task correctly.
By (3) of Property~\ref{prop:iv} and 
(2) of Properties~\ref{prop:common-coin} and~\ref{prop:data-transfer},
$P$ is a $k$-resilient equilibrium.
\end{proof}