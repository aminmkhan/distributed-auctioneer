

% Section: RELATED WORK
\section{Related Work}
\label{sec:related-work}

A vast literature has addressed the problem of allocating resources
between providers and users following an auction approach~\cite{Nisan2001, %Zhou2014 
Zheng2014Star, Wang:12,Zhang:13, Popa2012, Niu2012,
Shi:14,  Zhang:14, Zhang2015Truthful}.
For instance, in~\cite{Zhang:14,Zhang2015Truthful}, the authors propose Vickrey–Clarke–Groves (VCG) mechanism~\cite{Nisan2001}
in auctions where only the users submit bids to the auctioneer,
achieving truthfulness and a tradeoff between maximal social welfare and computational efficiency;
Zheng et al.~\cite{Zheng2014Star} propose a variant of the McAfee mechanism 
to tackle a similar problem in a double auction (providers also submit bids),
where they achieve truthfulness and budget balance.
To the best of our knowledge, none of these works addressed these problems
in the absence of a trusted auctioneer.

The problem of simulating the behaviour of a trusted entity
in an environment with only rational players has been approached
in the literature of distributed systems\,\cite{Halpern:04,Abraham:06,Abraham:13,Afek:14}.
In~\cite{Halpern:04,Abraham:06}, the authors addressed the
particular problems of secret sharing and multiparty computation
assuming the existence of a trusted mediator, and then studied
conditions under which it is possible to simulate the mediator
through a distributed protocol. Abraham et al.\,\cite{Abraham:13}
devised $k$-resilient equilibria solutions for the problem of leader election.
Afek et al.\,\cite{Afek:14} proposed a building blocks approach
for devising distributed $k$-resilient implementations, and
used this approach in combination with ideas from~\cite{Abraham:13}
to address the problems of consensus and renaming.
None of these works devised distributed protocols for simulating 
the role of an auctioneer in an auction.
