
% Section: CONCLUSION
\section{Conclusion and Future Work}
\label{sec:conclusion}

Resource allocation is a fundamental problem in networked systems and the design of auction mechanisms that can provide properties such as
truthfulness, budget balance, and maximal social welfare have been extensively studied in the literature. 
These works assume a centralised trusted auctioneer that can faithfully execute the allocation algorithm. 
Unfortunately, many networked systems of today, such as ``clouds of clouds'', edge clouds, and community networks, among others, lack a central trusted point of control. 
In this paper we have addressed the theoretical and practical challenges that need to be overcome to bridge this gap. 
More precisely, we have proposed a novel distributed framework for devising Nash equilibria 
distributed simulations of the auctioneer that are resilient to asynchrony and coalitions. 
Furthermore, our framework allows for the parallelisation of the allocation algorithm, 
leveraging the distributed nature of the simulation, which is of paramount practical importance given that, 
in many allocation algorithms, achieving maximal social welfare is computationally intensive. 
We have devised implementations of the framework in a realistic testbed of one of the largest community networks deployed today, 
and have gathered experimental evidence that the overhead of the emulation 
is not significant even in the cases the allocation algorithm cannot be parallelised, 
and brings substantial gains in the case parallelisation is possible. 
This shows that our approach can be used as a building block to implement resource allocation in decentralized networks.
