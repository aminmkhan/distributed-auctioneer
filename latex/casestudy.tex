
\section{Case Study}
\label{sec:instances}

In this paper, we use community networks as a concrete example of one of the many application areas for the distributed simulation of a trusted auctioneer.

\subsection{Community Networks}
Community networks are a bottom-up social initiative that provide Internet and networking services using off-the-shelf network hardware and the open, unlicensed wireless spectrum,
with optical fibre in some cases,
based on self-servicing and self-management by the users~\cite{Selimi2015Cloud}.  %~\cite{Baig2015Guifi}  ~\cite{Braem2013}
Many of such initiatives have proven quite successful, and 
	include AWMN in Greece, FunkFeuer in Austria, Freifunk in Germany, Ninux in Italy,
and Guifi.net which is one of the largest community networks in the wold and connects more than 28,000 locations.

Community networks are based on the principle of reciprocal sharing and most of their users are moved by altruistic principles. 
However, as any other human organisation, these networks are not immune to overuse, free-riding, or under-provisioning, specially in scenarios where users may have motivations to compete for scarce resources. 
In this paper, we consider the concrete problem of bandwidth reservation on the gateways that connect the community network to the Internet. 
In particular, we consider that, as in most community networks, the subset of users that offer gateway services to the Internet is smaller than the complete set of users. 
Users that are not at the gateways may be interested in reserving bandwidth in these gateways for Internet access.
If the available bandwidth at each gateway is not enough to satisfy the demand, 
one needs to implement some arbitration mechanism that optimally and fairly allocate resources, and at the same time maximises the resource utilisation and the provider's revenue. 

In the context of our model (\S~\ref{sec:model}), \emph{providers} are the owners of the gateways, and have direct access to the Internet,
while \emph{bidders} have no direct access to the Internet and rely on these gateways.
The role of \emph{auctioneer} can be taken by one of the providers, or a chosen third party,
or in the case of our proposed distributed auctioneer, simulated by some providers in a distributed fashion.
We consider \emph{resource} to be the bandwidth external to the network available at the gateways.
Note that we are not considering the bandwidth on the links internal to the community network,
because network is operated in a shared manner, 
and this bandwidth is collectively owned by the community.

\subsection{Auctions For Bandwidth Allocation}

We now show how our framework can be applied to two different bandwidth allocation problems in the context of community networks. For that purpose, we resort to two different 
algorithms that have been proposed in the literature to solve bandwidth allocation of bidders in providers. These 
algorithms rely on standard and double auctions respectively, and have different computational properties: the double auction algorithm 
provides an example of a graph with only one task that is not computationally intensive, such that decomposing its execution into 
parallel tasks does not provide a performance gain;  the standard auction algorithm provides a graph with multiple computationally 
intensive tasks that can be parallelised. Later in the paper, we will use these examples to evaluate the performance of implementations
of the framework. We use the double auctions example to measure a worst-case overhead of executing all building blocks of the framework 
compared to an execution with a centralised trusted auctioneer, and we use the standard auctions example to show that the improvements of
parallelisation can outweigh the added overhead when the execution time is dominated by computation.

\subsubsection{Double Auction}
\label{sec:instances-double-auction}
Consider an auction where each provider has a limited bandwidth to be allocated to multiple users,
and each user has a demand of bandwidth that may be satisfied by multiple providers.
Both the users and the providers declare in their bids the value given to a unit
of allocated bandwidth. An allocation gives the amount of bandwidth of each user
allocated in each provider. We want to ensure truthfulness in expectation and budget balance.
For this purpose, we use the algorithm $\A$ of~\cite{Zheng2014Star},
which provides the above properties at the expense
of social welfare. The idea is to order the providers by increasing
value and to order the users by decreasing value.
Then, users are allocated by their order to the providers using the water-filling method:
the maximum amount of bandwidth of each user is allocated
to the first available provider without exceeding its capacity,
and any unsatisfied demand of that user is allocated to the following providers using the same method.
Since the most computationally intensive task of this algorithm
is sorting, in most practical settings there is no performance gain in parallelising the execution of $\A$.
Instead, every provider executes $\A$ locally and outputs the result.
Hence, we never need to invoke the data transfer building block.


\subsubsection{Standard Auction}
\label{sec:instances-vcg}

%% ALGORITHM
\begin{algorithm}[tbp]
	\caption{Standard auction allocator}
	\label{alg:bandwidth-allocation-randomised}
	\small

	\begin{algorithmic}[1]
		\State Task 1: Calculate the allocation solution $x$
		\For {Each subset $S$ of bidders in parallel}
			\State Task 2.$S$: calculate payment $p_j$ of every $j \in S$
		\EndFor
		\State Task 3: Collect the outputs of each task with the data transfer and output $(x,\vec{p})$
	\end{algorithmic}

\end{algorithm}

Consider a variation of the double auction where providers
do not send bids and each bidder can only have its bandwidth demand
allocated in a single provider. Here, we aim for truthfulness
in expectation, maximal social welfare, and computational efficiency.
It is well known that a VCG mechanism can be used to provide the first two guarantees.
The difficulty is that determining the maximal
social welfare is in general an NP Hard problem,
which conflicts with the goal of computational efficiency.
To address this issue, we use the algorithm of~\cite{Zhang2015Truthful}
which adapts the VCG mechanism to achieve a tradeoff
between the two conflicting requirements. Specifically,
\cite{Zhang2015Truthful}~offers a $(1-\epsilon)$ approximation
of maximal social welfare for an arbitrarily small $\epsilon$,
while terminating in polynomial time according to smoothed analysis.

Interestingly, the randomised algorithm proposed
in~\cite{Zhang2015Truthful} has the potential for parallelisation.
In a course manner, the algorithm can be divided into three steps, depicted
in Algorithm\,\ref{alg:bandwidth-allocation-randomised}. The first
step derives an approximately optimal allocation of users to providers.
This step is hard to parallelise effectively in a distributed system,
so we run it in a single sequential task.
The second step calculates the payments for each user based on the result of the first step.
This step is computationally intensive and the payments for each user can be computed
independently and, therefore, can be easily parallelised.
The final step gathers all intermediate results to produce the output.
In our implementation, the first and third steps are executed by all providers.
In the second step, we group the providers into $c$ groups, each containing at least $k+1$
providers. Each group is assigned the computation of the payments
of a subset of $n/c$ users. Then, all providers of a group
execute the data transfer block to transfer the resulting payments to all providers.
