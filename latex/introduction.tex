

% Section: INTRODUCTION
\section{Introduction}
\label{sec:introduction}

Most distributed systems have limited resources that need to be shared by many nodes. 
For instance, in a network there is limited bandwidth that needs to be allocated to multiple nodes. 
In cloud applications, virtual machines (VMs) need to be allocated to different cloud users. 
Resource allocation is, therefore, a key problem in distributed systems.

Distributed resource allocation is particularly challenging when nodes
operate under different spheres of control and may not be willing to cooperate. 
Namely, a resource allocation strategy that assumes that all nodes 
execute a given algorithm may break if nodes 
may extract benefits by deviating from the expected behaviour.
Many examples of this problem can be found in the literature. 
The works of~\cite{Lee2007} and~\cite{Xiao2013} illustrate how
a network user may attempt to monopolize the bandwidth utilization 
if it has the opportunity. 
There is evidence that programmers can instrument their code 
to get an unfair advantage of several Unix schedulers~\cite{Grosu2005}. 
In shared infrastructures, like Grid systems, participating
users try to maximise their own usage to the detriment of 
the others~\cite{Lai2004}.
Dynamic wireless spectrum allocation suffers from unfair manipulation~\cite{Zhou2008}.
Social cloud computing~\cite{Caton2014}, 
and cooperative computing systems like BitTorrent~\cite{Liu2010}
suffer when users act selfishly in consuming resources.


An approach that has emerged as a viable alternative for the problem
above is to use economic models to address resource allocation, in
particular by resorting to auction systems~\cite{Riley1981}. 
As a result, an extensive
literature exists on the use of several types of auctions to perform
resource allocation in distributed systems~\cite{Niyato2013, Lai2004, Waldspurger1992}. 
In particular, the advent of the cloud computing model, where many
clients may compete for the resources managed by one or more
providers, has spurred the usage of different auction mechanisms in a
variety of resource allocation proposals for the
cloud~\cite{Popa2012, Niu2012, Wang:12, 
Zhang:13, Shi:14, Zheng2014Star, Zhang:14, Zhang2015Truthful}.

In these approaches, users are modelled as non-cooperative rational
players who are willing to pay for using resources or get paid for
providing those resources. Specifically, users declare to an
auctioneer the preference for different allocations of resources, and
the auctioneer executes some auction mechanism to derive an allocation
between users and resources that maximises social welfare (preferences of users for the allocation),
and the payments to be performed or
received by each user. The aim is to obtain an allocation with a social welfare as close as
possible to the optimal while ensuring truthfulness from the
users, such that they do not have incentives to lie about their
bids. In addition to maximal social welfare and truthfulness, other guarantees may be provided,
including computational efficiency and budget balance (the
payments made by the users outweigh the payments received).

These works assume that the auctioneer is trusted.  Unfortunately,
this is an unreasonable assumption in many of today's fully
decentralized systems, where all nodes are either resource consumers,
resource providers, or both. 
In this case, there is no natural candidate that 
can be trusted by all other nodes to run the auction
algorithm faithfully, given that any node may extract some
benefit by perturbing the auction result. In some sense, all current
distributed systems that rely on auctions to perform resource
allocation are not fully decentralized, because they depend on a
unique central point of control, which is the trusted node that runs
the auction algorithm. 
%TODO @amin: Luis, please review this statement.
This leads to the observation that there is
still a substantial gap that needs to be bridged to apply these
results in fully decentralized settings.

This paper bridges this gap by proposing a framework of distributed
protocols that allows multiple resource providers in a decentralized
distributed system to simulate the role of the auctioneer.  Such simulation
raises significant challenges both from the theoretical and practical
points of view. From the theoretical perspective, although there is a
vast literature of distributed fault-tolerant algorithms, with very
few exceptions (for instance,~\cite{Abraham:13}), these works do not
consider rational behaviour. From the practical perspective, the
distribution of the auctioneer may incur in additional overhead. Our
paper addresses both concerns. First, we prove that our distributed
simulations are sound from a game theoretical
perspective. Specifically, we show that the simulations are
$k$-resilient (ex post) equilibria~\cite{Abraham:13}, i.e., Nash
equilibria resilient to asynchrony and coalitions of providers of size
at most $k$. Second, our protocols leverage the distributed nature of
the resulting virtual trusted entity to parallelise the resource
provisioning algorithm, compensating for the additional costs imposed
by coordination. 

We have implemented a prototype of our protocols and evaluated the
resulting system in a real instance of a fully decentralized network,
namely an experimental testbed for community networks~\cite{Braem2013}.  
Different from the traditional
business-focused model applied by telecommunication operators, each
user in a community network is an owner of a portion of the total
infrastructure, which builds the mesh network.
Community networks are an interesting application scenario for
our results  because they lack a central point of control and make
extensive use of resource sharing. In particular, we consider the
concrete problem of bandwidth reservation on the gateways that connect
the community network to the Internet.  

Even though in our example we only focus on bandwidth
reservation, the fundamental problems at hand are general and emerge
every time shared resources need to be allocated to users in a set of
providers, such as for allocation of processing and memory resources
as virtual machines in public clouds~\cite{Zhang2015Truthful},
assignment of frequencies in secondary wireless spectrum
markets~\cite{Zhou2008}, and resource scheduling in grid and cloud
infrastructures\cite{Lai2004}.


In summary, the main contributions of this paper are the following:

\vspace{0.3mm}
- We propose a framework for devising distributed protocols executed
among a decentralized set of service providers that correctly simulate the
auctioneer in a family of resource allocation auctions.

\vspace{0.3mm}
- We show that every implementation of the framework is a $k$-resilient (ex post) equilibrium.
These implementations also tolerate users that send invalid bids.

\vspace{0.3mm}
- We show that it is possible to leverage the distributed nature of our framework to parallelise implementations,
mitigating scalability issues of purely centralised solutions.

\vspace{0.3mm}
- We implemented instances of the framework and reported the results from its deployment on the actual Guifi nodes.

\vspace{0.3mm} 

The rest of the paper is organised as follows.  In
Section~\ref{sec:related-work} we present related work.
Section~\ref{sec:model} provides the system model.  In
Section~\ref{sec:design}, we present the framework for simulating the
auctioneer and in Section~\ref{sec:instances} we illustrate its
applicability to the execution of two different auction mechanisms,
with different computational properties, in the context of community networks.  In
Section~\ref{sec:evaluation} we discuss results from the experiments.
Section~\ref{sec:conclusion} concludes the paper.
